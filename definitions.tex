\documentclass{article}

\usepackage{fancyhdr}
\setlength{\headheight}{12pt}
\pagestyle{fancy}

%set-up page dimentions
\usepackage[top=1.5 in, bottom = 1.5 in ,left = 1.5 in, right = 1.5in]{geometry}

\setlength{\parskip}{12pt}  % 12 pt = space between paragraphs
\setlength{\parindent}{0pt} % 0 pt  = indentation
\usepackage{amsmath}
\usepackage{amsthm}
\usepackage{amssymb}
\usepackage{amsthm}
\usepackage{ifthen}
\usepackage{latexsym}
\usepackage{graphicx}
\usepackage{graphics}
\usepackage{psfrag}
\usepackage{graphpap}
\renewcommand{\P}{\text{P}}
\newcommand{\C}{\text{C}}


% Allows hyperlinks if compiled with pdflatex
\usepackage{hyperref}
\hypersetup{colorlinks}
\usepackage{color}
\definecolor{darkred}{rgb}{0.5,0,0}
\definecolor{darkgreen}{rgb}{0,0.5,0}
\definecolor{darkblue}{rgb}{0,0,0.5}
\hypersetup{ colorlinks,
                linkcolor=darkblue,
                filecolor=darkgreen,
                urlcolor=darkblue,
                citecolor=darkblue }
%hyperlink example is: \href{http://www.google.com}{google}


%This environment establishes the header
\newenvironment{header}[2]{
\rhead{#1}
\lhead{#2}
\rfoot{}
\renewcommand{\headrulewidth}{0pt}  %use for a solid line at head
\renewcommand{\footrulewidth}{0pt}  %use for solid line at bottom
}



\newcommand{\natnums}{{\mathbb N}}
\newcommand{\algnums}{{\mathbb A}}
\newcommand{\rationals}{{\mathbb Q}}
\newcommand{\reals}{{\mathbb R}}
\newcommand{\norm}[1]{\left|\left|#1\right|\right|}
\newcommand{\unorm}[1]{{\left|\left|#1\right|\right|_u}}
\newcommand{\scriptR}{\mathcal{R}}
\newcommand{\scriptP}{\mathcal{P}}
\newcommand{\taggedP}{\dot{\mathcal{P}}}
\newcommand{\scriptQ}{\mathcal{Q}}
\newcommand{\taggedQ}{\dot{\mathcal{Q}}}


\begin{document}
Start filling in and adding!  Github location is: \href{http://github.com/ottocode/realAnalysis}{here at my account}.

\section{Definitions}
\begin{description}
\item Intervals \label{interval}\hfill \\
	An interval is a set \(I\) where if \(a,b\in I\) then \(\forall c\) such that \(a<b<c,\; c\in I\).

\item Compact \label{compact}\hfill \\
	A set is compact if it is \hyperref[closed]{closed} and \hyperref[bounded]{bounded}.

\item Closed \label{closed}\hfill \\
There are a number of definitions for \(A\subseteq \mathbb{R}\) being closed:
\begin{itemize}
\item A set is closed if it contains all of its limit points.
\item \(\forall \{a_n\} \) such that \(a_n \in A\), if \(\lim_{n\to\infty}a_n=a\), then \(a\in A.\)
\item \(A\) contains all of its accumulation points.
\item The complement of \(A\) is open (int \(\mathcal{U}\).)
\end{itemize}

\item Bounded \label{bounded}\hfill \\
\begin{itemize}
\item The set \(A\) is bounded if \(\exists M\in\mathbb{R}\) such that \(\forall a\in A, |a|<M.\)
\item The function \(f:A\to \mathbb{R}\) is bounded if \(\exists M\in\mathbb{R}\) such that \(\forall a\in A, |f(a)|<M.\)
\end{itemize}

\item Bounded Function Set \label{setofboundedfunctions}\hfill \\
If \(A\subseteq\mathbb{R}\), \(b(A)\) is the set of all bounded functions from \(A\to\mathbb{R}\).

\item Cauchy \label{cauchy}\hfill\\
A sequence $\{A_n\}$ is said to be cauchy if and only if $\forall \epsilon>0$, $\exists N\in \natnums$ such that $n,m>N$ $$|A_n-A_m|<\epsilon$$

\item Continuous Function Set \label{setofcontinuousfunctions}\hfill \\
If \(A\subseteq\mathbb{R}\), \(c(A)\) is the set of all continuous functions from \(A\to\mathbb{R}\).

\item Convergence of $f_n$ uniformly on $A$ \label{f_nuniformconvergence}\hfill \\
If $f_n:A\to \reals$ and $f:A_0\to \reals$ we say $f_n \displaystyle{\to^u} A$, $f_n$ converges uniformly on $A$, if $||f_n-f||_u\to 0$

\item Differentiable Function \label{differentiable}\hfill\\
If $g$ is differentiable at $a$ with $f'(a)=L$ if $\displaystyle{lim_{h\to 0}|\frac{f(a+h)-f(h)}{h}|=L}$.

or

$\forall \epsilon>0$, $\exists \delta>0$ such that if $|h|<\delta$ then $\displaystyle{|\frac{f(a+h)-f(h)}{h}|-L<\epsilon}$

\item Partition \label{partition}\hfill \\
\(\mathcal{P}=(x_0, x_1, \dots, x_n)\) is said to be a partition of \([a,b]\) if \(a=x_0, x_1, \dots, x_n=b\).

\item Tagged Partition \label{taggedpartition}\hfill \\
\(\dot{\mathcal{P}}=(x_0, x_1, \dots, x_n)\) is said to be a tagged partition of \([a,b]\) if it is a partition and \(t_i\in [x_{i-1}, x_i]\).

\item Riemann Sum \label{riemannsum}\hfill \\
If \(f\in b[a,b]\) and \(\dot{\mathcal{P}}\) is a tagged partition, then \[s(f,\dot{\mathcal{P}}) = \sum_{i=1}^n f(t_i)(x_i-x_{i-1})\] is called a Riemann sum of \(f\) for tagged partition \(\dot{\mathcal{P}}\).

\item Riemann Integrable \label{riemannintegrable}\hfill \\
Let \(f\in b[a,b]\), we say \(f\) is Riemann Integrable, written \(f\in \mathcal{R}[a,b]\) if \(\exists L\in\mathbb{R}\) such that \(\forall \epsilon>0,\;\exists\delta_\epsilon>0\) such that 
\[|s(f, \dot{\mathcal{P}}) - L| < \epsilon \text{ whenever }||\dot{\mathcal{P}}|| < \delta_\epsilon.\]
Thus we can say that 
\[\int_a^b f = L.\]

\item Norm of a Partition \label{partitionnorm}\hfill \\
If \(\mathcal{P}\) is a partition of \([a,b]\) then 
\[||\mathcal{P}|| = \max_{1\leq i\leq n}\{x_1-x_0, x_2-x_1, \dots, x_i- x_{i-1},\dots, x_n-x_{n-1}\}\]
 is called the "norm" of \(\mathcal{P}.\)

\item Indicator Function\label{indicator}\hfill \\


\item Lipschitz Continuous\label{Lipschitz}\hfill \\

\item Pointwise Convergence of Functions\label{pointwiseConv}\hfill \\
$\forall x \in E$ and $\forall \epsilon > 0$, $\exists N \in N$ so that $\forall n\geq N$ we
have$ |fn(x)- f(x)| < \epsilon$

\item Uniform Convergence of Functions\label{uniformConv}\hfill \\
if for every $\epsilon > 0$ $\exists N$ so that $n \geq N$
implies that $|fn(x) -f(x)| < \epsilon$ holds for all $x\in E$.
    
\item Uniform Norm\label{unorm}\hfill \\
Let $f$ be bounded on $A$ then the uniform norm, $||f||_u$, $||f||_\infty$, $||f||_A$, $$||f||_u=\sup\{|f(x)|:x\in A\}$$ is the smallest upper bound for $f$ on $A$. 

\item Cauchy wrt Uniform Norm\hfill \\
    Let \(\{f_n\}\) be a sequence with \(f_n\in b(A) \forall n\).  \\
    We say \(\{f_n\}\) is Cauchy wrt the uniform norm if \(\forall\;\epsilon >0 \,\exists\,N\in\natnums\) such that 
    if \(n,m\geq N,\) then
    \[\norm{f_n-f_m}_u < \epsilon.\]
    
\end{description}

\section{Theorems}
\begin{enumerate}
\item Bolzano-Weirstrauss Theorem \label{BWT}\hfill \\

\item Heinel -Borel Theorem \label{HBT}\hfill \\

$K\subseteq \reals$ is compact if and only if every open cover has a finite subcover.

\item Intermediate Value Theorem \label{IVT}\hfill \\

\item Riemann Integrable Function Uniqueness \label{RIFU}\hfill \\
Suppose $f\in \scriptR[a,b]$ then $\displaystyle{\int_a^b}f$ is unique.

\item Squeeze Theorem \label{Squeeze}\hfill \\

\item Cauchy Criterion for Integration  \label{CauchyCritInt}\hfill \\

\item All continuous functions are Riemann Integrable \label{continuousEqR}\hfill \\

\item Monotonic functions with the reals as an image are Riemann Integrable \label{monotonicEqR}\hfill \\

\item Integral of an interval equals the sum of the integrals between two "connected" intervals\hfill \\
        Suppose \(f:[a,b]\to\reals\) and \(c\in(a,b)\), then \(f\in\scriptR[a,b]\) iff \(f\in\scriptR[a,c]\) and \(f\in\scriptR[c,b]\).

\item Fundamental Theorem of Calculus, part I\label{FTC1}\hfill \\

\item Fundamental Theorem of Calculus, part II\label{FTC2}\hfill \\
If $f\in \scriptR[a,b]$ and $c_0\in[a,b]$ and $F(x)=\displaystyle{\int_{c_0}^x f}$ and $f(c)\in c[a,b]$ then $F$ is differentiable at $c$ and $F'(c)=f(c)$.

\item Lipschitz Continuity Criteria\label{LipschitzCont}\hfill \\

    \item Change of Variables (substitution)\label{usub}\hfill \\

    \item Integration by Parts\label{intbyparts}\hfill \\

    \item Sequences that imply continuity\label{seqimpc}\hfill \\
        If \(\{f_n\}\) is a sequence of functions \(f_n\in c(A)\) and \(f_n\stackrel{u}{\to} f\) on \(A\), then \(f\in c(A)\).

    \item More with sequences,  I guess these don't all need names... or labels.\hfill \\
        Suppose for all \(n\; f_n\in\scriptR[a,b]\) and \(f_n\stackrel{u}{\to} f\) on \([a,b]\).  Then \(f\in\scriptR[a,b]\) and \(\displaystyle{\lim_{x\to\infty}\int_a^bf_n = \int_a^b f.}\).

    \item \hfill \\
        Every Cauchy sequence in \(b(A)\) converges in \(b(A)\) with respect to \(\norm{\cdot}_u\).  
        That is, \(b(A)\) is complete wrt \(\norm{\cdot}_u.\)
        
      \item Triangle inequality for uniform norm\label{triinequality}\hfill \\
      $\forall x\in A$ where $||f||_u+||g||_u$ is an upper bound for $\{|f(x)|\}$ $$||f+g||_u\leq ||f||_u+||g||_u$$

\end{enumerate}




\section{Examples}
Add examples in class or other examples you may think of here

\section{Homework}
Add homework problems here.  Not necessarily every one, but the ones in particular that would make great examples.
The homework problems that give a theorem, or lemma, or other super-cool results, feel free to put in the Theorems section.

\end{document}
